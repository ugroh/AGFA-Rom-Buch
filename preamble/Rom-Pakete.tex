%% -- Die Pakete für RomSeminar
%% -- Stand: 2023/02/02
%% -- Literatur: H. Voss, Einführung in LaTeX (7. Auflage) 2022 
%% -- Muss mal sehen, wie ich dieses besser machen kann
%% --

%% -- Mathematik
\usepackage{%
	,amsmath
	,amssymb
	,amsthm} 
\usepackage[tbtags]{mathtools}

%% -- Theoreme etc. ohne Nummerierung
%% -- 
\newtheoremstyle{rom}
{3pt}% ⟨Space above ⟩ 
{3pt}% ⟨Space below ⟩ 
{}% ⟨Body font ⟩
{}% ⟨Indent amount ⟩2
{\itshape}% ⟨Theorem head font⟩
{}% ⟨Punctuation after theorem head ⟩
{.5em}% ⟨Space after theorem head ⟩ 
{}% 
\theoremstyle{rom}
	\newtheorem*{theorem}{Theorem}
	\newtheorem*{thm}{Theorem}
	\newtheorem*{proposition}{Satz}
	\newtheorem*{prop}{Satz}
	\newtheorem*{corollary}{Korollar}
	\newtheorem*{cor}{Korollar}
	\newtheorem*{lemma}{Lemma}

%% -- Symbole
%% --
\newcommand{\C}{\mathbb{C}}
\newcommand{\R}{\mathbb{R}}
\newcommand{\Q}{\mathbb{Q}}
\newcommand{\Z}{\mathbb{Z}}
\newcommand{\N}{\mathbb{N}}

%% -- Sonstiges
%% --


%% -- Aufzählungen
%% -- 
\usepackage[%
	,inline			% für Aufzählungen im Text = Sternvariante
	,shortlabels	% damit \begin{enumerate}[(i)] funktioniert
	]{enumitem}
	
%% -- tikz
\usepackage{tikz}
\usetikzlibrary{positioning,arrows}

%% -- Pakete mit Referenz
%% -- 
\usepackage[newcommands,footnotes,raggedrightboxes]{ragged2e}
	
\usepackage{% Voss, LaTeX - 6.3 - Seite 227ff					
	,array 			   	
	,booktabs
	,tabularx
	,wrapfig
	,longtable}
	
\RequirePackage{% Voss, LaTeX - 8.6 - Seite 340ff					
	,caption 			   	
	,floatrow
	,subcaption}

\RequirePackage{% 						
	,multicol	% Voss, LaTeX - 5.15.1 - Seite 215	
	,parallel	% Voss, LaTeX - 5.15.2 - Seite 216
	}
	 	   	
\RequirePackage{%
	,graphicx	% Voss, LaTeX - 5.10.2 - Seite 174
	,wrapfig	% Voss, LaTeX - 5.10.3 - Seite 177
	,cutwin		% Voss, LaTeX - 5.10.3 - Seite 178
	}

%% -- Wenn schon, dann muss man es auch definieren
%% -- 
\usepackage{nicefrac}

%% -- Zum Testen
%% -- 
\usepackage{blindtext,lipsum}

%% -- Für Rahmen
%% -- 
\usepackage{mdframed}
%% -- infobox
\newmdenv{infobox} %% \begin{infobox} .. \end{infobox}

%% -- Querverweise
%% -- 
\usepackage[nospace]{varioref}	% \vref
% -----------------------------------------------------------
\RequirePackage[%            	%  texdoc hyperref;                                                                                                         
	,colorlinks=true    		%  Farbige Links true                                                           
	,urlcolor=blue       		%                                                              
	,citecolor=blue      		%                                                              
	,linkcolor=blue      		%    
	,breaklinks= true                                                     
	]{hyperref}          		%% \href{xterner-Link}{Bezeichnung-in-Doku}
% -----------------------------------------------------------
\usepackage{cleveref}

