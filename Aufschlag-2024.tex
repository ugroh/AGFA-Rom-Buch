%% %%%%%%%%%%%%%%%%%%%%%%%%%%%%%%%%%%%
%% Für Was: Romseminar 2024 	
%% Stand: 	2023/04/12
%% %%%%%%%%%%%%%%%%%%%%%%%%%%%%%%%%%%
\documentclass[%					 
	,ngerman				
	,DIV = 13
	,titlepage=false		
	,twoside=false	 
	]{scrbook}  

\usepackage{./preamble/Rom-Buch}
   
%% -- Titelseite
%% --
\title{Romseminar 2024}
\subtitle{Krisen -- Konflikte -- Katastrophen \\
aus Sicht von Mathematik und Informatik}
\date{}
\renewcommand{\dictumwidth}{0.45\textwidth}
%% --
%% -- Leerzeilen belassen
%% --

\begin{document} 
\maketitle
\thispagestyle{empty}
\dictum[Max Frisch 1911--1991]{Krise ist ein produktiver Zustand. Man muss ihr nur den Beigeschmack der Katastrophe nehmen.}   
%%
\bigskip        

Finanz-Krise, Generationen-Konflikt, Umwelt-Katastrophe und Krieg sind die K-Wörter, die unser Leben prägen und uns zu Entscheidungen oder Handlungen zwingen. 
Sie können zerstören, aber auch Impulse zu Innovationen und neuen Ideen geben.
In diesem Seminar untersuchen wir, auf welche Weise Mathematik und Informatik involviert sind: als Ursache, als Objekt oder gar als Lösung und Hilfe. So können wir fragen:
%
\begin{enumerate}[$\bullet$]

\item
Welche Krisen und Konflikte -- etwa um ihre Grundlagen -- gab und gibt es in unseren (und in anderen) Wissenschaften? 

\item
Wie kann man Krisen (etwa in der Ökonomie oder Ökologie) durch Mathematik und Informatik verstehen und bewältigen?

\item
Können wir mit Mathematik Konflikte lösen, zum Beispiel bei der gerechten Verteilung einer Erbschaft oder bei der Nutzung von Naturressourcen?

\item
Welche Rolle spielte oder spielt Mathematik und Informatik in Kriegshandlungen? Könnten diese Disziplinen und die Praktizierenden in Kriegszeiten unschuldig sein?

\end{enumerate}
%
Diese und andere Themen werden im Romseminar 2024 angesprochen, je nach Präferenzen der Teilnehmenden. Dieses Seminar bietet die Möglichkeit, über den Tellerrand des eigenen Studienfachs hinauszuschauen und Themen von allgemeiner Relevanz (vor dem Hintergrund unserer Fächer) zu diskutieren. 
Darüber hinaus geht es auch darum, Präsentation, Rhetorik und Diskussion in einem fachlichen Kontext zu üben.

Im Laufe des Wintersemesters 2023/24 werden wir zunächst den Themenkomplex durch gemeinsame Lektüre, Impulsreferate und Diskussionen erschließen. 
Alle Teilnehmenden sollen bis Ende Dezember ihre eigenen Projekte gefunden haben (allein oder mit Partnern). 
Dies wird schließlich in der gemeinsamen Exkursionsphase in Rom (25. Februar bis 2. März 2024) vorgestellt und diskutiert. 
Durch ein vielfältiges Begleitprogramm lassen wir uns auch zu sonst nicht zugänglichen Orten dieser \enquote{Ewigen Stadt} inspirieren.

Voraussetzung für die Teilnahme am Seminar ist die Bereitschaft, sich mit der Thematik intensiv auseinanderzusetzen. 
Die Veranstaltung eignet sich für Studierende ab dem ersten Semester.

 \end{document}
 